\documentclass{article}
\usepackage[utf8]{inputenc}
\usepackage{amsmath, amssymb, amsthm}

\title{Übung 2}
\author{Pascal Diller, Timo Rieke}

\setcounter{secnumdepth}{0}

\begin{document}

\maketitle

\section{1}
\subsection{(i)}
\[R_1^{-1} = \{(z,x),(y,z)\}\]
a
\subsection{(ii)}
\subsection{(iii)}

\section{2}
\subsection{(i)}
Die Relation ist asymmetrisch, da aus $f(0) < g(0)$ folgt, dass $f(0) \ngtr g(0)$ \\
Die Relation ist transitiv, da \\
$\implies$ strikte Ordnung
\subsection{(ii)}
Die Ordnung ist total, da zwischen den zwei Werten $f(0)$ und $g(0)$ immer gilt: $f(0) < g(0)$

\section{3}
\subsection{(i)}
R ist reflexiv, da wenn $x = y$ gilt: $f(x) = f(y)$ \\
R ist symmetrisch, da wenn $x = y$ auch gilt: $y = x$ \\
R ist transitiv, dan wenn $x = y$ und $y = z$, auch $x = z$ \\
$\implies$ Äquivalenzrelation
\subsection{(ii)}

\section{4}
\subsection{(i)}
\subsection{(ii)}
\subsection{(iii)}

\section{5}

\end{document}
